\chapter{Results}
The experiment does not confirm the theory.
In general, the literature values do not lie within their respective error boundaries.
Both methods approximately yield the same error limits and values.
The considered errors turn out to be large in relation to their values.
However, keeping in mind that the error limits in method two can be minimized with increasing measurement time and series, one can obtain better certainty on the quantities to be determined.

Nevertheless, resolution time could be determined to lie within the calculated error boundaries, as the resolution time was defined to be \SI{200}{\ns}.

The fact that all resolting quantities do not resemble their respective literature values suggests that either the experiment was performed incorrectly, or the used equipment is faulty.
%Considering \autoref{fig:e_spectrum} the indistinct peaks and blurred slopes in the energy spectrum encourages the assumption that the second explanation might be more likely to be true.\todo{Just looking for excuses... :D} % not really, who gives a sh*t about the exact energy anyways? the equipment is not trimmed to be linear in energy

The experiment can be improved by increasing the measurement time and checking the equipment.
