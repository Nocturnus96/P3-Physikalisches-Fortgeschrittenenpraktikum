\chapter{Data}
The gathered data will be evaluated using two different methods.
In the first method, all reduced event counts for each angle are added up and used for calculating coefficients $a_2$ and $a_4$,
whereas in the second method, the data is analyzed for each individual measurement series.

\section{False Coincidences}
Suppose two individual decays randomly happen at the same time, hence are not correlated.
Using the described detector setup would result into measuring a false coincidence.
To compensate for this, a time offset is applied to the measured event counts later on in the analysis.
This way, true coincidences are generally excluded and only false coincidences remain, which can be subtracted from the total measured coincidence counts.
\autoref{tab:false} shows measured false coincidences.

In the following, false coicidences are denoted with $N_\text{F}$.

\section{Background}\label{sec:bg}
Background radiation data can be seen in \autoref{tab:background}.
As data suggests, background radiation will not have a significant effect on the data as it is several orders of magnitude weaker than the examined effect.

Denoting background radiation counts with $N_\text{bg}$.

\section{Distance Correction}
As already mentioned in \autoref{sec:geom}, distance correction must be applied to the event counts.
In subsequent sections, distances will have been corrected already.

\section{Reduced Event Counts}
To account for temperature drifts and thresholds, all coincidence counts must be divided by their individual associated counts after backgrounds and false coincidences are subtracted
\begin{equation}\label{eq:red}
	R(\theta)=\frac{N_\text{c}-N_\text{cbg}-N_\text{F}}{(N_1-N_\text{bg})(N_2-N_\text{bg})}.
\end{equation}

Assuming a poisson distribution, the errors of event counts are their square root, which gives
\begin{align*}
	\sigma_\text{R} &= \sqrt{\left(\pdb{R}{N_\text{c}}\cdot\sigma_{N_\text{c}}\right)^2
	+ \left(\pdb{R}{N_\text{F}}\cdot\sigma_{N_\text{F}}\right)^2
	+ \left(\pdb{R}{N_1}\cdot\sigma_{N_2}\right)^2
	+ \left(\pdb{R}{N_2}\cdot\sigma_{N_2}\right)^2} \\
	&=\frac{1}{(N_1-N_\text{bg})(N_2-N_\text{bg})}
	\cdot\sqrt{N_\text{c} + N_\text{F} + \left(\frac{N_\text{c}-N_\text{F}}{N_1-N_\text{bg}}\right)^2\cdot N_1
	+ \left(\frac{N_\text{c}-N_\text{F}}{N_2-N_\text{bg}}\right)^2\cdot N_2},
\end{align*}
for the propagated error of reduced event counts, where background radiation is negligably small as discussed in \autoref{sec:bg}.

\section{Evaluation: Method One}
Adding all reduced event counts for each angle and propagating their corresponding errors like
\begin{equation*}
	\sigma_{R_\theta}=\sum_{i=1}^{6}\sigma_\text{R,i}^2,
\end{equation*}
where $i$ denotes the number of the measurement series, yields the values
\begin{itemize}
	\item $R_\theta (90)=\num{3.640(79)e-7}$
	\item $R_\theta (135)=\num{3.303(80)e-7}$
	\item $R_\theta (90)=\num{3.953(83)e-7}$
\end{itemize}
for the sum of reduced coincidences per angle.

Using formulae \ref{eq:A} and \ref{eq:B}, the auxiliary quantities $A$ and $B$ are calculated as
\begin{align*}
	A &= \num{0.907(30)} \\
	B &= \num{1.086(33)},
\end{align*}
while their corresponding errors propagate like
\begin{align*}
	\sigma_\text{A} &= \sqrt{\left(\pdb{A}{R_\theta (90)}\cdot\sigma_{R_\theta(90)}\right)^2 + \left(\pdb{A}{R_\theta (135)}\cdot\sigma_{R_\theta(135)}\right)^2} \\
	\sigma_\text{B} &= \sqrt{\left(\pdb{B}{R_\theta (90)}\cdot\sigma_{R_\theta(90)}\right)^2 + \left(\pdb{B}{R_\theta (180)}\cdot\sigma_{R_\theta(180)}\right)^2}.
\end{align*}

Finally, the required correlation function coefficients and anisotropy can be calculated.
Utilizing equations \ref{eq:a2}, \ref{eq:a4} and \ref{eq:aniso} yields the values
\begin{alignat*}{3}
	a_2 &= \num{-0.456(123)} &&\Rightarrow \text{(relative deviation from theoretical value: 464.96\%)}\\
	a_4 &= \num{0.542(136)}  &&\Rightarrow \text{(relative deviation from theoretical value: 1200.87\%)} \\
	An  &= \num{0.086(33)}   &&\Rightarrow \text{(relative deviation from theoretical value: 48.61\%)},
\end{alignat*}
while their errors propagate like
\begin{align*}
	\sigma_\text{$a_2$} &= \sqrt{\left(\pdb{a_2}{A}\cdot\sigma_\text{A}\right)^2 + \left(\pdb{a_2}{B}\cdot\sigma_\text{B}\right)^2} \\
	&=\sqrt{\left(4\cdot\sigma_\text{A} \right)^2 + \left(-1\cdot\sigma_\text{B} \right)^2} \\
	\sigma_\text{$a_4$} &= \sqrt{\left(\pdb{a_4}{A}\cdot\sigma_\text{A}\right)^2 + \left(\pdb{a_4}{B}\cdot\sigma_\text{B}\right)^2} \\
	&=\sqrt{\left(-4\cdot\sigma_\text{A} \right)^2 + \left(2\cdot\sigma_\text{B} \right)^2} \\
	\sigma_\text{An} &= \sigma_\text{B}.
\end{align*}

\section{Evaluation: Method Two}
