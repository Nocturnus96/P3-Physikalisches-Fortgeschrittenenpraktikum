\chapter{Procedure}\label{chap:explaining-shit}
\section{Neutron Source}
In the following experiment, an Am-Be-source is used.
Neutrons are produced when the $\alpha$-particles are absorbed by a low-atomic-weight isotope.
Here, Americium is used as an $\alpha$-source and beryllium is used as a target.
The reaction equation can be written as
\begin{equation*}
	\ce{^{9}Be} + \alpha \rightarrow n + \ce{^{12}C} + Q.
\end{equation*}
Typical neutron radiation energies are in a \si{MeV}-range.
No significant amount of $\alpha$ radiation leaves the source, since \ce{^{241}Am} and \ce{^{9}Be} are alloyed together and each $\alpha$-particle is expected to hit a beryllium nucleus within its reach.
The energy spectrum is continuous, as a result of the isotropy of radiation.

\section{Relaxation Length of Fast Neutrons}\label{sec:relaxa}
Taking the logarithm of \autoref{eq:relax} and using the definition of the relaxation length $\lambda$, it follows that
\begin{equation*}
	\log(r^2\cdot\phi(r)) = -\frac{r}{\lambda} + \text{const.}.
\end{equation*}
This equation can be used to determine $\lambda$ by measuring $\phi$ for varying distances $r$ and fitting the data to this model.

However, this relation is only true for primary neutrons.
Experimentally, primary and secondary neutrons are difficult to distinguish.
This problem is addressed by using a \ce{^{10}B} detector, which is only sensitive for thermal neutrons.
This sounds contradictory, since we want to detect primary neutrons.
However, the emitted neutrons primarily interact with the protons of the water molecules, the cross section increases with descreasing energy.
This means that the neutrons have a small mean free path inside the moderator.
Thermalization takes place near the location of interaction.
As a result, the distribution of thermal neutrons follows the distribution of primary neutrons.

Nevertheless, this method only yields an approximation of the true relaxation length of fast neutrons inside water and can only be interpreted qualitatively.

\section{Diffusion Length}
Applying the same transformations to \autoref{eq:relax} as in \autoref{sec:relaxa} yields
\begin{equation*}
	\log(r\cdot\phi(r)) = -\frac{r}{L} + \text{const.}.
\end{equation*}
This equation is only applicable for point sources, which is not possible for thermal neutrons.
That is why the so called \ce{Cd} differential method is used:
Two measurements are carried out, one measuring the flux $\phi_0(r)$ of the neutrons emitted by the unshielded source and one measuring the flux $\phi_\text{Cd}(r)$ with the neutron source enclosed by a cadmium case.
This cadmium case filters out any thermal neutrons with a kinetic energy below \SI{5}{meV}.
The first measurement detects neutrons which were thermalized at any point in space, the second only detects neutrons that were thermalized outside the cadmium shield.
Subtracting the data of the second from the data of the first measurement yields the flux of neutrons that were thermalized within the cadmium shield, creating a virtual thermal neutron point source.
