\setcounter{chapter}{-1}
\chapter{Prep}
\section{Neutron Interactions}\label{sec:inter}
There are three processes in which fast neutrons may interact with nuclei:
\begin{itemize}
	\item \textbf{Absorption:} The nucleus accepts the neutron through emission of quanta of corresponding energies.
	The cross section for this process reaches its maximum at low energies.
	\item \textbf{Elastic Scattering:} The total kinetic energy of the system is conserved.
	Since the target usually is at rest in the lab frame, the neutrons lose part of their kinetic energy and are scattered into different directions.
	The energy transfer is dependent on the target's mass and takes its maximum value for the proton's mass.
	Here too, the maximal cross section is achieved with low energies.
	\item \textbf{Inelastic Scattering:} The neutron interacts with the nucleus and the total kinetic energy of the system is changed, often activating the nucleus.
	The neutrons usually lose more energy in this process than by elastic scattering, albeit the cross sections generally taking smaller values.
\end{itemize}

\section{Neutron Flow}\label{sec:flow}
To describe the neutron radiation field, we can define a quantity $n(\vec{r}, \vec{\Omega}, E)$, the differential neutron density.
It describes the amount of neutrons at a location $\vec{r}$ in space with unit energies around $E$ per unit solid angle around $\vec{\Omega}$.
Like this, we may define a radiant flux
\begin{equation}\label{eq:flux}
	\phi(\vec{r}) = \int_E \int_\Omega n(\vec{r}, \vec{\Omega}, E)\cdot v(E)\cdot \,dE \,d\Omega,
\end{equation}
where $v(E)$ denotes the absolute value of the neutrons' velocity, which is subject to their energies $E$.
Defining the neutrons' mean velocity as
\begin{equation*}
	\bar{v} = \frac{\phi(\vec{r})}{\int_E \int_\Omega n(\vec{r}, \vec{\Omega}, E)\cdot \,dE \,d\Omega}
\end{equation*}
\autoref{eq:flux} can be written more compactly
\begin{equation*}
	\phi(r) = n(r)\cdot \bar{v} \qquad \left[\phi\right]=\si{\per\meter\squared\per\second},
\end{equation*}
which can be understood as the amount of neutrons that pass through the unit area per time unit.

\section{Relaxation Length}
Consider a point source emitting neutrons.
Its radiant flux is given by
\begin{equation}\label{eq:relax}
	\phi(r) = \frac{Q_0}{4\pi r^2}\cdot e^{N\sigma_\text{t}\cdot r}
\end{equation}
where $Q_0$ denotes the source strength, which is defined by the amount of emissions per time unit.
$N\sigma_t$ is the total linear absorption coefficient, given by the sum of the absorption coefficients for all possible interactions (see \autoref{sec:inter})
\begin{equation*}
	N\sigma_t = N\cdot(\sigma_\text{el} + \sigma_\text{a}),
\end{equation*}
where $\sigma_\text{el}$ and $\sigma_\text{a}$ denote the cross sections for elastic scattering and absorption.
As the cross section for inelastic scattering is negligably small.
$N$ is the density of target nuclei.

With this knowledge, we can define the quantity
\begin{equation*}
	\lambda = \frac{1}{N\cdot\sigma_\text{t}},
\end{equation*}
which we call \textbf{relaxation length}, the mean path length of neutrons before being scattered or absorbed.

\section{Thermalization of Fast Neutrons}
Neutrons emitted by a source with kinetic energies higher than those of their target nuclei will always lose energy through interactions.
After a fair amount of collisions, their excess kinetic energy will have been completely transferred into the target.
The neutrons now are in thermal equillibrium with their environment, they are \textbf{thermalized}.
Their velocities now obey a Maxwell distribution with the mean velocity of
\begin{equation*}
	\bar{v}_\text{th}=\sqrt{\frac{2kT}{m_\text{n}}},
\end{equation*}
with a respective energy of
\begin{equation*}
	E_\text{th}=\frac{m\bar{v}_\text{th}^2}{2} = kT,
\end{equation*}
which are \SI{25}{\meV} at room temperature.
This knowledge will be of use for the measurement of the \textbf{diffusion length} of thermal neutrons.

\section{Diffusion Length}
The fundamental goal of diffusion analysis is to find energy and spatial distributions for a given setup of neutron source and target.
This can be accomplished by considering the \textbf{transport equation}, a partial differential equation for the differential neutron density $n(\vec{r}, \vec{\Omega}, E)$, discussed in \autoref{sec:flow}.
Substantially, it is a continuity relation, which describes the neutron balance for a given differential neutron density.
The transport equation has no solutions in a closed form, however, it can be solved approximately under a few conditions.
Considering stationary solutions which are not dependent on the energy.
The only processes allowed are scattering and absorption, which is deemed to be very weak ($\sigma_\text{a}\ll \sigma_\text{s}$).
Naturally, these considerations do not allow for the description of the thermalization process itself, since it is no monoenergetic process but can be applied to already thermalized neutrons.

With these conditions the transport equation simplifies to
\begin{alignat*}{3}
	D\cdot\Delta\phi(r) - N\sigma_\text{a}\cdot\phi(r) + S(r) &= 0 \qquad D&&=\frac{1}{3N\sigma_\text{a}}\\
	\Delta\phi(r) - \frac{1}{L^2}\cdot\phi(r) + \frac{S(r)}{D} &= 0 \qquad L&&=\frac{D}{N\sigma_\text{a}},
\end{alignat*}
where $L$ is called the \textbf{diffusion length}.
The solution of the equation above with the boundary condition $\lim_{r\rightarrow\infty}\phi = 0$ is
\begin{equation}\label{eq:sol}
	\phi(r) = \frac{Q_0}{4\pi\cdot D}\cdot \frac{e^{-r/_L}}{r},
\end{equation}
which can be used as a sample form for the measurement of $L$.

\chapter{Experiment}
\section{Neutron Source}
In the following experiment, an Am-Be-source is used.
Neutrons are produced when the $\alpha$-particles impinge upon a low-atomic-weight isotope.
Here, Am is used as an $\alpha$-source which affects Be.
The reaction equation can be written as
\begin{equation*}
	\ce{^{9}Be} + \alpha \rightarrow n + \ce{^{12}C} + Q.
\end{equation*}
Typical neutron radiation energies are in a \si{MeV}-range.
No significant amount of $\alpha$ radiation is to be expected, since \ce{^{241}Am} and \ce{^{9}Be} are finely mixed and each $\alpha$-particle is expected to hit a beryllium nucleus within its reach.
The energy spectrum is continuous, as a result of the isotropy of radiation.

\section{Relaxation Length of Fast Neutrons}\label{sec:relaxa}
Taking the logarithm on both sides of \autoref{eq:relax} and using the definition of the relaxation length $\lambda$, it follows that
\begin{equation*}
	\log(r^2\cdot\phi(r)) = -\frac{r}{\lambda} + \text{const.}.
\end{equation*}
This equation can be used to measure $\lambda$ by measuring $\phi$ for varying distances $r$ and fitting the data to this model.

However, only primary neutrons are permitted to be measured using this method.
Unfortunately, primary and secondary neutrons are difficult to distinguish.
This problem is addressed by using the \ce{^{10}B} detection method, which is only sensitive for thermal neutrons.
This sounds contrdictory, since we want to detect primary neutrons.
However, the emitted neutrons primarily interact with the protons of the water molecules, the cross section increases with descreasing energy.
This means that the neutrons have a small mean free path inside the moderator.
Thermalization takes place near the location of interaction.
As a result, the distribution of thermal neutrons follows the distribution of primary neutrons.

Nevertheless, this method only yields an approximation of the true relaxation length of fast neutrons inside water and can only be interpreted qualitatively.

\section{Diffusion Length}
Applying the same transformations as in \autoref{sec:relaxa} to \autoref{eq:relax} yields
\begin{equation*}
	\log(r\cdot\phi(r)) = -\frac{r}{L} + \text{const.}.
\end{equation*}
This equation is only applicable for point sources, which is not possible for thermal neutrons.
That is why the so called \ce{Cd} differential method is used:
Two measurements are carried out, one measuring the flux $\phi_0(r)$ of the emitted neutrons by the source and one measuring the flux $\phi_\text{Cd}(r)$ with the neutron source being enclosed by a cadmium case.
This cadmium case filters out any thermal neutrons up to \SI{5}{meV}.
The first measurement detects a mixture of fast and thermal neutrons, the second only detects fast neutrons.
Subtracting the data of the second from the data of the first measurement hence yields a virtual thermal neutron point source.
\end{document}
