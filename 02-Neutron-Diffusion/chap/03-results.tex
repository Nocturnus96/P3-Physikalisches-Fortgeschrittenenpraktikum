\chapter{Results}\label{chap:explaining-shitty-results}%probably never referenced, but still totally appropriate
Most data points lie outside the calculated error bounds as can be seen in \autoref{fig:figure-i-didnt-label-yet} and \autoref{fig:figure-i-didnt-label-yet-2}.
A source of systematic error is the finite size of the neutron source and the detector.
This error can be reduced by measuring at higher distances and, to compensate for the much lower count rates, extending the aquisition time.
There are no definite literature values to compare the results to, but the results of most research papers\footnote{\url{http://nvlpubs.nist.gov/nistpubs/jres/51/jresv51n4p203_A1b.pdf}}\footnote{\url{http://www.tandfonline.com/doi/abs/10.13182/NSE62-A26058}}\footnote{\url{https://www.jstage.jst.go.jp/article/jaesj1959/2/9/2_9_542/_pdf}} and other lab groups lie close to the determined lengths.
