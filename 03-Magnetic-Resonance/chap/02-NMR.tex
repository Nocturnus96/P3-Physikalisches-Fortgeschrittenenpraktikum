\chapter{NMR}
\todo{maybe mix up the first words a bit}
The experiment is conducted with a commercial NMR analyzer, the \texttt{Varian EM 360-A}, which uses an analog chart recorder to store the spectrum.
The probe is placed in a constant, homogenous magnetic field which can be trimmed to improve accuracy.
Pressurized air is used to spin the probe, effectively making the field more homogenous. %really want to draw a parallel to a microwave stove
The perturbing frequency is controlled by the position of the chart recorder's head.
The sweep range and the center frequency can be selected via rotary switches, the sweep duration can be selected to improve resolution or speed up the measurement.
To increase time efficiency, a quick scan is used to find peaks.
When recording the final spectrum, the sweep time is increased around the peaks and decreased where accuracy is not important.

\section{Calibrating the Magnetic Field}
Several small coils are placed around the permanent magnet, allowing fine adjustment of the magnetic field in the sample.
A vial of water is used as the sample and the field is trimmed as per the instructions provided.

%nope, we did not, but isn't there something in the lab instruction? I recall, they put in the spectrum for water as well.

\section{Acetic Acid (sample 1)}
The spectrum shows two distinctive peaks, corresponding to the hydrogen atoms in the methyl and carboxyl groups.
The peaks can be identified by their intensity, the methyl group has a higher intensity.
In this spectrum, the peak is twice as high as the carboxyl peak instead of the predicted ratio of three, this is possibly due to clipping when the recorder head hit the end stop.
The frequencies are \SI{2.18}{ppm} for the methyl group and \SI{11.5}{ppm} for the carboxyl group.
The high shift of the carboxyl group is due to the high electronegativity of oxygen, pulling the electrons close to the carboxyl group and shielding the single hydrogen nucleus from the external field.

\section{Booze (sample 2)}
In booze (also referred to as ethanol) \todo{Do not forget to change this! :D}, there are three distinctive groups containing hydrogen atoms.

\begin{center}
	\begin{tabular}{cSSc}
		\toprule
		functional group&	{chemical shift $\delta$}&	{coupling constant $J$}& multiplet\\
		\midrule
		hydroxy (\ce{OH})&	5.44& {singlet}\\
		alkyl (\ce{CH_2})&	3.81&	0.123&	4\\
		methyl (\ce{CH_3})&	1.22&	0.117&	3\\
		\bottomrule
	\end{tabular}
\end{center}

The hydroxy and alkyl lines again are shifted strongly due to the electronegative oxygen atom.
Also the lines of the alkyl and methyl groups are split into a quadruplet and triplet, with very similar coupling constants. \todo{Explain the splitting.}

\section{Ethanol and Hydrochloric Acid (sample 8)}

The sample that also contains hydrochloric acid is missing the hydroxy line and instead shows a peak that corresponds to the acid.
This is due to NMR being a slow process on atomic scale.
The hydrogen atom of the hydroxy group is in an equilibrium reaction between \ce{HCl} and \ce{OH}.
This means that the difference in the population between higher and lower energy states, which is necessary for net absorption of radiation to occur, cannot form.

The chemical shifts of the two carbon groups are higher, this can be explained by the electronegative chlorine attracting electrons from the ethanol and thus leaving the nuclei shielded less.

\begin{center}
	\begin{tabular}{cSSc}
		\toprule
		functional group&	{chemical shift $\delta$}&	{coupling constant $J$}& multiplet\\
		\midrule
		hydroxy (\ce{OH})&\\
		alkyl (\ce{CH_2})&	3.84&	0.128&	4\\
		methyl (\ce{CH_3})&	1.28&	0.123&	3\\
		\ce{HCl}&	7.5& {singlet}\\
		\bottomrule
	\end{tabular}
\end{center}

\section{1- and 2-Propanol (sample 9 and 3)}

Despite sharing the same chemical formula, the two isomers contain different chemical groups.
Isopropanol contains two equivalent methyl groups which show up as one peak and one \ce{CH} group, whereas the 1-propanol has one methyl group and two \ce{CH_2} groups which are not equivalent and show up as two separate peaks.
The two separate peaks of the \ce{CH_2} groups are due to the different neighbouring groups, the ''B'' group, which is closer to the oxygen atom, has a higher shift than group ''A''.

\begin{center}
	n-PrOH (groups are sorted by position in molecule)\\
	\begin{tabular}{cSSc}
		\toprule
		functional group&	{chemical shift $\delta$}&	{coupling constant $J$}& multiplet\\
		\midrule
		methyl (\ce{CH_3})&	0.97&	0.116&	3\\
		alkyl A (\ce{CH_2})&	1.63&	0.116&	5\\
		alkyl B (\ce{CH_2})&	3.75&	\\
		hydroxy (\ce{OH})&	5.53&	\\
		\bottomrule
	\end{tabular}

	i-PrOH\\
	\begin{tabular}{cSSc}
		\toprule
		functional group&	{chemical shift $\delta$}&	{coupling constant $J$}& multiplet\\
		\midrule
		2x methyl (\ce{CH_3})&	1.28&	0.103& 3\\
		alkyl (\ce{CH})&	4.19&	0.109&	5\\
		hydroxy (\ce{OH})&	5.28&	\\
		\bottomrule
	\end{tabular}
\end{center}

\section{Methyl(methylthio)methylsulfate (sample 10)}
The compound contains three detectable groups.
Two dissimilar methyl groups show up as separate peaks, where the group next to the sulfinyl group ''A'' has a weaker shift than the group next to the sulfide ''B'', as the oxygen attracts electrons away from methyl group ''A'', decresing the shielding effect.
The \ce{CH_2} group shows a shift similar to previous samples.

\begin{center}
	\begin{tabular}{cSS}
		\toprule
		functional group&	{chemical shift $\delta$}&	{coupling constant $J$}\\
		\midrule
		methyl A (\ce{CH_3})&	2.51\\
		methyl B (\ce{CH_3})&	2.87\\
		alkyl (\ce{CH_2})&	4.21\\
		\ce{HCl}&	7.5& {singlet}\\
		\bottomrule
	\end{tabular}
\end{center}
