\chapter{ESR}

\section{Exercise 1: Intro-Experiment}
A passive, resonant circuit consisting of a coil and variable capacitor and the ESR probe are placed close to each other.
The AC voltage across the variable capacitor is measured while varying the ESR probe's frequency.
As expected, while both circuits entered into resonance, a voltage peak can be measured across the resonant circuit.
This means that the resonant circuit obviously drains energy from the ESR probe's electromagnetic field, thus creating a voltage trough at the probe's voltage which is visible in the spectrum.

\section{Exercise 2: Electron Spin Resonance}
A DPPH-sample is inserted into the ESR probe and an external magnetic field is generated by a Helmholtz-coil.
The head of the ESR probe is placed at the coil's center point to ensure that the field is as homogenous as possible.
The probe's resonance peaks are measured for three different ESR coils.
Increasing the external magnetic field's amplitude causes the adjacent peaks in the spectrum to approach each other in a manner which leaves the resonant voltages constant.
Cycling through the available coils changes the external fields amplitude at a fixed supply voltage, which has the same effect as described above.
Lower inductivities mean lower external magnetic fields and vice versa.

\section{Exercise 3: External Field Dependence}
