\chapter{ESR}

\section{Exercise 1: Intro-Experiment}
A passive, resonant circuit consisting of a coil and variable capacitor and the ESR probe are placed close to each other.
The AC voltage across the variable capacitor is measured while varying the ESR probe's frequency.
As expected, while both circuits entered into resonance, a voltage peak can be measured across the resonant circuit.
This means that the resonant circuit obviously drains energy from the ESR probe's electromagnetic field, thus creating a voltage trough at the probe's voltage which is visible in the spectrum.

\section{Exercise 2 and 3: External Field Dependence}
A DPPH-sample is inserted into the ESR probe and an external magnetic field is generated by a Helmholtz-coil.
The head of the ESR probe is placed at the coil's center point to ensure that the field is as homogenous as possible.
The probe's resonance peaks are measured for three different ESR coils.
Increasing the external magnetic field's amplitude causes the adjacent peaks in the spectrum to approach each other in a manner which leaves the resonant voltages constant.
Cycling through the available coils changes the external fields amplitude at a fixed supply voltage, which has the same effect as described above.
Lower inductivities mean lower external magnetic fields and vice versa.

Upon removing the sample from the ESR probe, the characteristic spectrum vanishes.
This means that the measured peaks actually originate from the sample's resonance.
Increasing the HF amplitude of the ESR probe increases the signal strength in the spectrum.
This behavior can be explained by the increasing number of spins that are excited inside of the sample.
Hence, for extraordinarily high amplitudes growth of the signal strength should be pushed into saturation.
This behavior cannot be observed as the amplitude available is limited by the ESR probe.

\section{Exercise 4: Measuring the g-Factor}
\subsection{Data}
\begin{table}[tbp]
	\begin{subfigure}{.5\textwidth}
		\caption[g-factor measurement: First coil]{\textbf{g-factor measurement: First coil ($f=13-30\si{\MHz}$)}}
		\label{tab:firstcoil}
		\begin{tabular}{S|SSS}
			\toprule
			{$f$ in \si{\MHz}}	&	{$U_1$ in \si{\mV}}	&	{$U_2$ in \si{\mV}}	&	{$\bar{U}$ in \si{\mV}} \\
			\midrule
			13	 & 	160	 & 	-96.6	&	128.3	\\
			18	 & 	210	 & 	-150	&	180	\\
			23	 & 	257	 & 	-200	&	228.5	\\
			28	 & 	300	 & 	-267	&	283.5	\\
			33	 & 	347	 & 	-307	&	327	\\
			\bottomrule
		\end{tabular}
	\end{subfigure}
	\hfill
	\begin{subfigure}{.4\textwidth}\small
		\caption[g-factor measurement: Second coil]{\textbf{g-factor measurement: Second coil ($f=30-75\si{\MHz}$)}}
		\label{tab:secondcoil}
		\begin{tabular}{S|SSS}
			\toprule
			{$f$ in \si{\MHz}}	&	{$U_1$ in \si{\mV}}	&	{$U_2$ in \si{\mV}}	&	{$\bar{U}$ in \si{\mV}} \\
			\midrule
			70	 & 	720	 & 	-666	&	693	\\
			80	 & 	806	 & 	-753	&	779.5	\\
			90	 & 	916	 & 	-866	&	891	\\
			100	 & 	1020	 & 	-966	&	993	\\
			110	 & 	1120	 & 	-1070	&	1095	\\
			120	 & 	1220	 & 	-1170	&	1195	\\
			130	 & 	1320	 & 	-1270	&	1295	\\
			\bottomrule
		\end{tabular}
	\end{subfigure}
\end{table}

	\begin{table}[tbp]
		\centering
		\caption[g-factor measurement: Second coil]{\textbf{g-factor measurement: Third coil ($f=30-75\si{\MHz}$)}}
		\label{tab:thirdcoil}
		\begin{tabular}{S|SSS}
			\toprule
			{$f$ in \si{\MHz}}	&	{$U_1$ in \si{\mV}}	&	{$U_2$ in \si{\mV}}	&	{$\bar{U}$ in \si{\mV}} \\
			\midrule
			70	 & 	720	 & 	-666	&	693	\\
			80	 & 	806	 & 	-753	&	779.5	\\
			90	 & 	916	 & 	-866	&	891	\\
			100	 & 	1020	 & 	-966	&	993	\\
			110	 & 	1120	 & 	-1070	&	1095	\\
			120	 & 	1220	 & 	-1170	&	1195	\\
			130	 & 	1320	 & 	-1270	&	1295	\\
			\bottomrule
		\end{tabular}
	\end{table}
The ESR spectrum is measured for all three inductivities while varying the HF frequency.
The external magnetic field is held at a constant level.
Measured data can be seen in tables \ref{tab:firstcoil}, \ref{tab:secondcoil} and \ref{tab:thirdcoil}.

As discussed in \autoref{sec:esr}, the system has to satisfy the resonance condition.
Rearranging \autoref{eq:resonance}, the g-factor can be determined by fitting the linear model
\begin{equation*}
	\nu=\underbrace{g\frac{\mu_\text{B}}{h}}_a \cdot B,
\end{equation*}
to the measured data.

The lab preparation provides the relation between the voltage and magnetic field generated by the Helmholtz coil
\begin{equation}\label{eq:extfield}
	B=\frac{U}{R}\cdot\SI{3.96}{\milli\tesla\per\ampere}.
\end{equation}
The voltage is measured across a $R=\SI{1.07}{\ohm}$ shunt resistor and is determined by averaging the absolute values of the voltages from both half-waves
\begin{equation*}
	U = \frac{\abs{U_1}+\abs{U_2}}{2}.
\end{equation*}

\subsection{Error Calculation}
Assuming an error of $\Delta U_i=\SI{50}{\mV}$, which complies with a half voltage-step of the used oscilloscope, the voltage uncertainty propagates like
\begin{alignat*}{2}
	\Delta U &= \sqrt{\sum^2_{i=1}\left(\pdb{U}{U_i}\cdot \Delta U_i\right)^2} \\
	&=\frac{1}{\sqrt{2}}\Delta U_i,
\end{alignat*}
which translates into the error of the magnetic field as
\begin{alignat*}{2}
	\Delta B &= \abs{\pdb{B}{U}}\cdot \abs{\Delta U_i} \\
	&= \frac{\SI{3.96}{\milli\tesla\per\ampere}}{\sqrt{2}R}\cdot\Delta U_i \\
	&= \SI{130.8}{\micro\tesla}.
\end{alignat*}

The uncertainty of the ESR frequency can be estimated as $\Delta\nu = \SI{100}{\kHz}$, which takes into account the fine adjustment means and age of the applicance.

\subsection{Results}
